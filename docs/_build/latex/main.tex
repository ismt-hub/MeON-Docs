%% Generated by Sphinx.
\def\sphinxdocclass{report}
\documentclass[a4paper,10pt,english]{report}
\ifdefined\pdfpxdimen
   \let\sphinxpxdimen\pdfpxdimen\else\newdimen\sphinxpxdimen
\fi \sphinxpxdimen=.75bp\relax

\PassOptionsToPackage{warn}{textcomp}
\usepackage[utf8]{inputenc}
\ifdefined\DeclareUnicodeCharacter
% support both utf8 and utf8x syntaxes
  \ifdefined\DeclareUnicodeCharacterAsOptional
    \def\sphinxDUC#1{\DeclareUnicodeCharacter{"#1}}
  \else
    \let\sphinxDUC\DeclareUnicodeCharacter
  \fi
  \sphinxDUC{00A0}{\nobreakspace}
  \sphinxDUC{2500}{\sphinxunichar{2500}}
  \sphinxDUC{2502}{\sphinxunichar{2502}}
  \sphinxDUC{2514}{\sphinxunichar{2514}}
  \sphinxDUC{251C}{\sphinxunichar{251C}}
  \sphinxDUC{2572}{\textbackslash}
\fi
\usepackage{cmap}
\usepackage[T1]{fontenc}
\usepackage{amsmath,amssymb,amstext}
\usepackage{babel}


\usepackage{amsmath,amsfonts,amssymb,amsthm}

\usepackage{fncychap}
\usepackage[,numfigreset=1,mathnumfig]{sphinx}
\sphinxsetup{hmargin={0.7in,0.7in}, vmargin={1in,1in},         verbatimwithframe=true,         TitleColor={rgb}{0,0,0},         HeaderFamily=\rmfamily\bfseries,         InnerLinkColor={rgb}{0,0,1},         OuterLinkColor={rgb}{0,0,1}}
\fvset{fontsize=\small}
\usepackage{geometry}


% Include hyperref last.
\usepackage{hyperref}
% Fix anchor placement for figures with captions.
\usepackage{hypcap}% it must be loaded after hyperref.
% Set up styles of URL: it should be placed after hyperref.
\urlstyle{same}

\usepackage{sphinxmessages}
\setcounter{tocdepth}{0}


        %%%%%%%%%%%%%%%%%%%% Meher %%%%%%%%%%%%%%%%%%
        %%%add number to subsubsection 2=subsection, 3=subsubsection
        %%% below subsubsection is not good idea.
        \setcounter{secnumdepth}{3}
        %
        %%%% Table of content upto 2=subsection, 3=subsubsection
        \setcounter{tocdepth}{2}

        \usepackage{amsmath,amsfonts,amssymb,amsthm}
        \usepackage{graphicx}

        %%% reduce spaces for Table of contents, figures and tables
        %%% it is used "\addtocontents{toc}{\vskip -1.2cm}" etc. in the document
        \usepackage[notlot,nottoc,notlof]{}

        \usepackage{color}
        \usepackage{transparent}
        \usepackage{eso-pic}
        \usepackage{lipsum}

        \usepackage{footnotebackref} %%link at the footnote to go to the place of footnote in the text

        %% spacing between line
        \usepackage{setspace}
        %%%%\onehalfspacing
        %%%%\doublespacing
        \singlespacing


        %%%%%%%%%%% datetime
        \usepackage{datetime}

        \newdateformat{MonthYearFormat}{%
            \monthname[\THEMONTH], \THEYEAR}


        %% RO, LE will not work for 'oneside' layout.
        %% Change oneside to twoside in document class
        \usepackage{fancyhdr}
        \pagestyle{fancy}
        \fancyhf{}

        %%% Alternating Header for oneside
        \fancyhead[L]{\ifthenelse{\isodd{\value{page}}}{ \small \nouppercase{\leftmark} }{}}
        \fancyhead[R]{\ifthenelse{\isodd{\value{page}}}{}{ \small \nouppercase{\rightmark} }}

        %%% Alternating Header for two side
        %\fancyhead[RO]{\small \nouppercase{\rightmark}}
        %\fancyhead[LE]{\small \nouppercase{\leftmark}}

        %% for oneside: change footer at right side. If you want to use Left and right then use same as header defined above.
        \fancyfoot[R]{\ifthenelse{\isodd{\value{page}}}{{\tiny Meher Krishna Patel} }{\href{http://pythondsp.readthedocs.io/en/latest/pythondsp/toc.html}{\tiny PythonDSP}}}

        %%% Alternating Footer for two side
        %\fancyfoot[RO, RE]{\scriptsize Meher Krishna Patel (mekrip@gmail.com)}

        %%% page number
        \fancyfoot[CO, CE]{\thepage}

        \renewcommand{\headrulewidth}{0.5pt}
        \renewcommand{\footrulewidth}{0.5pt}

        \RequirePackage{tocbibind} %%% comment this to remove page number for following
        \addto\captionsenglish{\renewcommand{\contentsname}{Table of contents}}
        \addto\captionsenglish{\renewcommand{\listfigurename}{List of figures}}
        \addto\captionsenglish{\renewcommand{\listtablename}{List of tables}}
        % \addto\captionsenglish{\renewcommand{\chaptername}{Chapter}}


        %%reduce spacing for itemize
        \usepackage{enumitem}
        \setlist{nosep}

        %%%%%%%%%%% Quote Styles at the top of chapter
        \usepackage{epigraph}
        \setlength{\epigraphwidth}{0.8\columnwidth}
        \newcommand{\chapterquote}[2]{\epigraphhead[60]{\epigraph{\textit{#1}}{\textbf {\textit{--#2}}}}}
        %%%%%%%%%%% Quote for all places except Chapter
        \newcommand{\sectionquote}[2]{{\quote{\textit{``#1''}}{\textbf {\textit{--#2}}}}}
    

\title{Sphinx format for Latex and HTML}
\date{May 02, 2020}
\release{}
\author{Meher Krishna Patel}
\newcommand{\sphinxlogo}{\vbox{}}
\renewcommand{\releasename}{ }
\makeindex
\begin{document}

\pagestyle{empty}

        \pagenumbering{Roman} %%% to avoid page 1 conflict with actual page 1

        \begin{titlepage}
            \centering

            \vspace*{40mm} %%% * is used to give space from top
            \textbf{\Huge {Sphinx format for Latex and HTML}}

            \vspace{0mm}
            \begin{figure}[!h]
                \centering
                \includegraphics[scale=0.3]{logo.jpg}
            \end{figure}

            \vspace{0mm}
            \Large \textbf{{Meher Krishna Patel}}

            \small Created on : Octorber, 2017

            \vspace*{0mm}
            \small  Last updated : \MonthYearFormat\today


            %% \vfill adds at the bottom
            \vfill
            \small \textit{More documents are freely available at }{\href{http://pythondsp.readthedocs.io/en/latest/pythondsp/toc.html}{PythonDSP}}
        \end{titlepage}

        \clearpage
        \pagenumbering{roman}
        \tableofcontents
        \listoffigures
        \listoftables
        \clearpage
        \pagenumbering{arabic}

        
\pagestyle{plain}
 
\pagestyle{normal}
\phantomsection\label{\detokenize{index::doc}}


Добродошли у документацију MeON апликације. У наредним поглављима објаснићемо концепт поменуте апликације, њене функционалности и рад у истој.


\chapter{Кратак преглед}
\label{\detokenize{uvod/index:id1}}\label{\detokenize{uvod/index::doc}}
MeON апликација је намењена евиденцији, праћењу и контроли процеса у области јавних набавки. Под процесом овде подразумевамо интерне и законске радње које се врше у фазама плана, поступка и реализације конкретне набавке.

Како би вам што детаљније приближили функционалности саме апликације, објаснићемо пре свега основна начела на којима MeON почива.


\section{Концепт}
\label{\detokenize{uvod/koncept:uvod}}\label{\detokenize{uvod/koncept:id1}}\label{\detokenize{uvod/koncept::doc}}
\noindent\sphinxincludegraphics[width=600\sphinxpxdimen]{{nab_dok}.png}

Када погледамо набавку кроз фазе И призму докумената видимо скуп докумената који настају у сваким од фазама процеса. Неки су јединствени, неки се понављају, неки су измене основних докумената итд. Ова документа су законски неопходна, наравно у зависности од типа, врсте набавке и врсте поступка набавке. Поред ових докумената, потреба је корисника да евидентирају и неке друге документе или информације везане за интерну организацију.

\noindent\sphinxincludegraphics[width=600\sphinxpxdimen]{{nab_hijer}.png}

Када се сви документи заједно са набавком групиши по хијерархији, добијамо скуп ентитета у одређеним релацијама (као на слици). Ово представља модел по коме МеОН апликација ради (објаснићу). Тако овде имамо набавку као главни ентитет, са одређеним атрибутима које евидентитамо и пратимо. Даље, испод имамо разне типове докумената (одлуке, обавештења, позиве, питања, конкурсну документацију, уговоре и фактуре). Сви ови ентитети имају сопствене атрибуте који се евидентирају и прате. Зашто је овај модел битан? Зато што се на основу ових веза и атрибута покрива целокупан процес процес у конкретном предузећу. Цела апликација је ограничена Моделом. Видећете да су сви менији дефинисани на основу дефиниције ентитета, гридови на основу поља итд… Ентитети, везе и атрибути су потпуно конфигурабилни. У преводу ово значи да можемо лако додавати нове атрибуте уколико се за то показе потреба, дефинисати нови статус итд…Типове података који се евидентирају ћу детаљно објаснити касније.

Друга страна медаље је да модел као такав мора бити потврдјен да ради и покрива потребе корисника. Моја екипа и ја смо развили овај модел на основу закона и искуства које смо стицали паралелно са развојем апликације у сарадњи са некима од вас. Свака помоћ око оцене модела је добродошла поготово од стране људи који су у овом послу дуго времена.


\chapter{MeON Приручник}
\label{\detokenize{prirucnik/index:meon}}\label{\detokenize{prirucnik/index::doc}}
У овом поглављу ћемо се детаљно бавити коришћењем апликације. Поглавље смо поделили у 5 главних делова:


\section{Навигација}
\label{\detokenize{prirucnik/navigacija:navigacija}}\label{\detokenize{prirucnik/navigacija:id1}}\label{\detokenize{prirucnik/navigacija::doc}}

\subsection{Главни мени}
\label{\detokenize{prirucnik/navigacija:id2}}

\subsection{Breadcrumbs}
\label{\detokenize{prirucnik/navigacija:breadcrumbs}}

\subsection{Мени са опцијама}
\label{\detokenize{prirucnik/navigacija:id3}}

\section{Евиденција}
\label{\detokenize{prirucnik/evidencija:evidencija}}\label{\detokenize{prirucnik/evidencija:id1}}\label{\detokenize{prirucnik/evidencija::doc}}

\subsection{Листе}
\label{\detokenize{prirucnik/evidencija:id2}}
\begin{sphinxadmonition}{warning}{Warning:}
Neque porro quisquam

Lorem ipsum dolor sit amet, consectetur adipiscing elit. Vivamus mattis commodo eros, quis posuere enim lobortis quis. Nullam ut tempus nibh.
\end{sphinxadmonition}


\subsection{Basic Specification}
\label{\detokenize{prirucnik/evidencija:basic-specification}}
Paragraphs contain text and may contain inline markup: \sphinxstyleemphasis{emphasis}, \sphinxstylestrong{strong emphasis}, \sphinxtitleref{interpreted text}, \sphinxcode{\sphinxupquote{inline literals}}, standalone hyperlinks (\sphinxurl{http://www.python.org}), external hyperlinks (\sphinxhref{http://www.python.org}{Python}), internal cross\sphinxhyphen{}references ({\hyperref[\detokenize{prirucnik/evidencija:example}]{\sphinxcrossref{example}}}), footnote references (%
\begin{footnote}[1]\sphinxAtStartFootnote
A footnote contains body elements, consistently
indented by at least 3 spaces.
%
\end{footnote}), citation references (\sphinxcite{prirucnik/evidencija:cit2002}), substitution references (), and \phantomsection\label{\detokenize{prirucnik/evidencija:inline-internal-targets}}inline internal targets.


\subsubsection{List}
\label{\detokenize{prirucnik/evidencija:list}}

\paragraph{Bullet lists:}
\label{\detokenize{prirucnik/evidencija:bullet-lists}}\begin{itemize}
\item {} 
This is a bullet list.

\item {} 
Bullets can be “*”, “+”, or “\sphinxhyphen{}“.

\end{itemize}


\paragraph{Enumerated lists:}
\label{\detokenize{prirucnik/evidencija:enumerated-lists}}\begin{enumerate}
\sphinxsetlistlabels{\arabic}{enumi}{enumii}{}{.}%
\item {} 
This is an enumerated list.

\item {} 
Enumerators may be arabic numbers, letters, or roman
numerals.

\end{enumerate}


\paragraph{Definition lists:}
\label{\detokenize{prirucnik/evidencija:definition-lists}}\begin{description}
\item[{what}] \leavevmode
Definition lists associate a term with a definition.

\item[{how}] \leavevmode
The term is a one\sphinxhyphen{}line phrase, and the definition is one
or more paragraphs or body elements, indented relative to
the term.

\end{description}


\paragraph{Field lists:}
\label{\detokenize{prirucnik/evidencija:field-lists}}\begin{quote}\begin{description}
\item[{what}] \leavevmode
Field lists map field names to field bodies, like
database records.  They are often part of an extension
syntax.

\item[{how}] \leavevmode
The field marker is a colon, the field name, and a
colon.

The field body may contain one or more body elements,
indented relative to the field marker.

\end{description}\end{quote}


\paragraph{Option lists, for listing command\sphinxhyphen{}line options:}
\label{\detokenize{prirucnik/evidencija:option-lists-for-listing-command-line-options}}\begin{optionlist}{3cm}
\item [\sphinxhyphen{}a]  
command\sphinxhyphen{}line option “a”
\item [\sphinxhyphen{}b file]  
options can have arguments
and long descriptions
\item [\sphinxhyphen{}\sphinxhyphen{}long]  
options can be long also
\item [\sphinxhyphen{}\sphinxhyphen{}input=file]  
long options can also have
arguments
\item [/V]  
DOS/VMS\sphinxhyphen{}style options too
\end{optionlist}


\subsubsection{Literal blocks:}
\label{\detokenize{prirucnik/evidencija:literal-blocks}}\begin{quote}
\begin{description}
\item[{if literal\_block:}] \leavevmode
text = ‘is left as\sphinxhyphen{}is’
spaces\_and\_linebreaks = ‘are preserved’
markup\_processing = None

\end{description}
\end{quote}


\subsubsection{Block quotes:}
\label{\detokenize{prirucnik/evidencija:block-quotes}}\begin{quote}

This theory, that is mine, is mine.

\begin{flushright}
---Anne Elk (Miss)
\end{flushright}
\end{quote}


\subsubsection{Simple Table}
\label{\detokenize{prirucnik/evidencija:simple-table}}

\begin{savenotes}\sphinxattablestart
\centering
\begin{tabulary}{\linewidth}[t]{|T|T|T|}
\hline
\sphinxstyletheadfamily 
Header row, column 1
&\sphinxstyletheadfamily 
Header 2
&\sphinxstyletheadfamily 
Header 3
\\
\hline
body row 1, column 1
&
column 2
&
column 3
\\
\hline
body row 2
&\sphinxstartmulticolumn{2}%
\begin{varwidth}[t]{\sphinxcolwidth{2}{3}}
Cells may span columns
\par
\vskip-\baselineskip\vbox{\hbox{\strut}}\end{varwidth}%
\sphinxstopmulticolumn
\\
\hline
\end{tabulary}
\par
\sphinxattableend\end{savenotes}


\subsubsection{Citation}
\label{\detokenize{prirucnik/evidencija:citation}}

\begin{fulllineitems}
\pysigline{\sphinxbfcode{\sphinxupquote{module=xml.xslt~class=Processor}}}
\end{fulllineitems}

\phantomsection\label{\detokenize{prirucnik/evidencija:example}}
The “\_example” target above points to this paragraph.


\subsubsection{Табела}
\label{\detokenize{prirucnik/evidencija:id6}}
Текст о гриду


\subsubsection{Претрага}
\label{\detokenize{prirucnik/evidencija:id7}}
Текст о претрази


\subsection{Детаљи}
\label{\detokenize{prirucnik/evidencija:id8}}
Текст о детаљима


\subsubsection{Типови поља}
\label{\detokenize{prirucnik/evidencija:id9}}

\subsubsection{Унос}
\label{\detokenize{prirucnik/evidencija:id10}}
Текст о уносу


\subsubsection{Измена}
\label{\detokenize{prirucnik/evidencija:id11}}

\subsubsection{Описи поља}
\label{\detokenize{prirucnik/evidencija:id12}}

\section{Календар}
\label{\detokenize{prirucnik/kalendar:kalendar}}\label{\detokenize{prirucnik/kalendar:id1}}\label{\detokenize{prirucnik/kalendar::doc}}

\subsection{Преглед}
\label{\detokenize{prirucnik/kalendar:id2}}

\section{Пошта}
\label{\detokenize{prirucnik/posta:posta}}\label{\detokenize{prirucnik/posta:id1}}\label{\detokenize{prirucnik/posta::doc}}

\subsection{Врсте информација}
\label{\detokenize{prirucnik/posta:id2}}

\subsection{Инбокс}
\label{\detokenize{prirucnik/posta:id3}}

\subsection{Чет}
\label{\detokenize{prirucnik/posta:id4}}

\section{Извештаји}
\label{\detokenize{prirucnik/izvestaji:izvestaji}}\label{\detokenize{prirucnik/izvestaji:id1}}\label{\detokenize{prirucnik/izvestaji::doc}}

\subsection{Типови извештаја}
\label{\detokenize{prirucnik/izvestaji:id2}}

\subsubsection{Ступчани извештаји}
\label{\detokenize{prirucnik/izvestaji:id3}}

\subsubsection{Табеларни извештаји}
\label{\detokenize{prirucnik/izvestaji:id4}}

\subsubsection{Пита извештаји}
\label{\detokenize{prirucnik/izvestaji:id5}}

\subsubsection{КПИ извештаји}
\label{\detokenize{prirucnik/izvestaji:id6}}

\subsection{Параметри}
\label{\detokenize{prirucnik/izvestaji:id7}}

\subsection{Извоз}
\label{\detokenize{prirucnik/izvestaji:id8}}

\chapter{Упутства за коринсике}
\label{\detokenize{uputstva/index:id1}}\label{\detokenize{uputstva/index::doc}}
У овом поглављу ћемо детаљно описати процедуре/сценарија за кориснике по њиховим улогама у организацији:


\section{Рола 1}
\label{\detokenize{uputstva/rola1:rola1}}\label{\detokenize{uputstva/rola1:id1}}\label{\detokenize{uputstva/rola1::doc}}

\subsection{Scenarii za rolu 1}
\label{\detokenize{uputstva/rola1:scenarii-za-rolu-1}}

\section{Рола 2}
\label{\detokenize{uputstva/rola2:rola2}}\label{\detokenize{uputstva/rola2:id1}}\label{\detokenize{uputstva/rola2::doc}}

\subsection{Scenarii za rolu 2}
\label{\detokenize{uputstva/rola2:scenarii-za-rolu-2}}
\begin{sphinxthebibliography}{CIT2002}
\bibitem[CIT2002]{prirucnik/evidencija:cit2002}
Just like a footnote, except the label is
textual.
\end{sphinxthebibliography}



\renewcommand{\indexname}{Index}
\printindex
\end{document}